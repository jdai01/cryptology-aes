\subsection{Strengths}

\subsubsection*{Cryptanalytic Strength and Security}

AES's greatest strength lies in its cryptographic security. As discussed in Section \ref{sec:security_issues}, its three key lengths make it highly resistant 
to brute-force attacks with existing and foreseeable computational resources. Even with theoretical advances such as quantum computing, 
AES-256 is believed to remain secure. Moreover, after more than two decades of intense public and academic scrutiny, no feasible attack 
has been discovered that compromises the full version of AES more efficiently than exhaustive key search.

\subsubsection*{Performance and Resource Efficiency}

AES excels in performance across software and hardware implementations. Its design allows efficient realisation in embedded devices, 
consumer hardware, and high-speed servers. In software, techniques such as T-Box-based lookup tables and processor instruction sets 
like Intel's AES-NI enable quick block cipher operations, with throughput often measured in hundreds of megabits or gigabits per second. 
In hardware, AES is compact enough for low-power chips (e.g., in smartcards, \Gls{RFID}, or \Gls{IoT} devices) but also scales up for multi-gigabit 
networking and storage applications via pipelining and parallelisation.

\subsubsection*{Global Standardization}

AES has earned a reputation for deep trust and widespread acceptance, thanks to its transparent and internationally recognized selection 
process. It is officially acknowledged in major standards and has been certified by the U.S. NSA for the protection of classified information 
up to the TOP SECRET level, specifically with 192 and 256-bit keys. This extensive regulatory endorsement guarantees that AES solutions remain 
compatible and consistent across various fields and organizations, from finance and business to government and healthcare \cite{cooper2025aes}.