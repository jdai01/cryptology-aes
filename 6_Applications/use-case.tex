\subsection{Common use case} 

\subparagraph{Wireless Security (Wi-Fi):} 
AES is commonly used together with Wi-Fi security protocols (such as WPA2 and WPA3) to encrypt data transmitted over wireless networks.
This is to ensure that sensitive information, like passwords and personal data, remains protected from unauthorised access \cite{cooper2025aes}.


\subparagraph{Encrypted Browsing (HTTPS):}
Websites use AES encryption within HTTPS protocols to ensure data transmitted between browsers and servers.
This encryption helps protect user information, such as login credentials and payment details, from interception by malicious actors \cite{cooper2025aes}.


\subparagraph{Virtual Private Networks:}
The job of a \gls{VPN} is to securely connect a user to another server online, only the best encryption can be considered so that the user's data will not be leaked.
The \glspl{VPN} that use AES with 256-bit keys include NordVPN, Surfshark, and ExpressVPN \cite{rimkiene2022aes}.


\subparagraph{File and Disk Encryption:}
Operating systems like Windows and macOS offer AES-based encryption options (e.g. BitLocker and FileVault) for securing entire hard drives or individual files.
This is particularly useful for safeguarding personal or sensitive business information stored on physical devices \cite{cooper2025aes}.


\subparagraph{Cloud Storage:}
AES encryption is essential for securing files stored in cloud environments.
Services like Google Drive, Dropbox, and others use AES to ensure that uploaded files remain confidential and protect against unauthorised access \cite{cooper2025aes}.


\subparagraph{Mobile Applications:}
Many mobile apps, especially those dealing with financial transactions or personal data, use AES encryption to secure data on devices and in transit.
This includes banking apps, social media platforms, and messaging apps, providing users with a peace of mind that their data is protected \cite{cooper2025aes}.


\subparagraph{Secure Messaging:}
Many encrypted messaging applications, like Signal and WhatsApp, use AES to secure messages end-to-end, ensuring that only the sender and recipient can read contents of their conversations \cite{cooper2025aes}.


\subparagraph{Password Managers:}
These are the programs that carry a lot of sensitive information.
Hence, password managers like LastPass and Dashlane include the important step of AES implementation \cite{rimkiene2022aes}.