\subsection{Limitations and considerations:}

Despite its many strengths, using AES in practice has several important challenges that can affect security 
and effectiveness of cryptographic solutions.

\subsubsection*{Key Management}

No symmetric cipher, including AES, can guarantee security if key management practices are poor. The entire 
method fundamentally depends on the secrecy, integrity, and lifecycle management of keys. Secure generation, 
distribution, rotation, storage (often using physical security modules), and destruction are critical.

Despite AES's mathematical strength, mistakes such as using weak keys, allowing key reuse, failing to rotate 
compromised keys, or transmitting keys over insecure channels can undermine the confidentiality of encrypted data.

\subsubsection*{Implementation Vulnerabilities}

AES's theoretical resilience is not immune to practical attacks targeting its real-world implementation. Side-channel 
attacks, including timing analysis, cache behaviour exploitation, power consumption monitoring, and electromagnetic 
radiation assessment, pose significant threats to the confidentiality of cryptographic key material. If the implementation 
is not carefully handled, these attacks can effectively disclose sensitive information. Mitigating these risks requires 
additional programming practices like constant-time implementations or mathematically secure curves.

\subsubsection*{Lack of Built-In Authentication}
AES offers confidentiality, but it does not include a way to verify that the encrypted data hasn't been tampered 
with. Data integrity and authenticity must be added using \Gls{MAC}s or by adopting combined authenticated encryption 
modes like \Gls{GCM} or \Gls{CCM} \cite{rfc4494}. Neglecting to supplement basic AES with integrity checks can expose the system to forgery 
and modification attacks.