\subsection{Security issues addressed} 
\label{sec:security_issues}

Over the years, several attacks have been proposed against the AES, 
including square attacks, impossible differential attacks, related-key attacks, and biclique attacks. 
However, none of these techniques broke the full-round AES under practical conditions.

The biclique attack currently represents the best-known classical cryptanalytic technique on the full AES. 
AES-128's complexity is approximately $2^{126}$, which is only marginally better than exhaustive key search. 
Similar minor improvements have been reported for AES-192 and AES-256. 
These theoretical results do not translate into feasible attacks in practice.

Algebraic approaches have also been explored by representing AES operations as polynomial equations. 
While these representations are mathematically elegant, 
no practical attack has emerged due to the infeasibility of solving the resulting systems on full-round AES.

In the quantum setting, Grover's algorithm reduces the complexity of brute-force search from $2^n$ to approximately $2^{n/2}$. 
However, recent quantum circuit analysis shows that this theoretical speedup does not compromise AES. 
The most advanced implementations estimate quantum attack complexities as $2^{156.3}$ for AES-128, $2^{221.6}$ for AES-192, 
and $2^{286.1}$ for AES-256. 

These figures, derived from the latest quantum resource evaluations by Jang et al.\cite{Jang2025}, 
indicate that AES remains secure despite quantum adversaries, 
given current and near-future technological capabilities.

All known classical and quantum attacks require unrealistic assumptions or apply only to reduced-round versions of AES. 
Consequently, AES continues to be considered secure for current and foreseeable post-quantum scenarios.