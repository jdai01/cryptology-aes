\subsection{The role of AES in data security}

\noindent AES has been widely used in practice since it was standardised by NIST more than 20 years ago. 
It is implemented in protocols and products such as TLS, IPsec, IEEE 802.11i, SSH, 
WhatsApp, Signal, hard disk encryption tools, and many others. 
AES is the most widely used cipher in the world today and has strongly influenced modern block cipher design. 
Its design includes the wide trail strategy, which shows how the linear layer helps resist statistical attacks. 
Many successful modern block ciphers use design elements originally introduced in AES. 
Currently, no analytical attack that is significantly better than brute force is known on AES.\newline

\noindent In addition to its strong theoretical foundations, 
AES is highly efficient in both software and hardware. 
It supports high-throughput implementations and benefits from dedicated CPU instructions such as AES-NI (Advanced Encryption Standard New Instructions), 
accelerating performance in many modern systems. 
Even advanced cryptanalysis techniques, such as biclique attacks, 
provide only negligible improvements over brute-force approaches.\newline

\noindent Through this unique combination of robust security, practical efficiency, and broad adoption, 
AES plays a central role in safeguarding digital data across various applications and platforms.