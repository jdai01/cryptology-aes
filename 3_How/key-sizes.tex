\subsection{Key sizes and rounds}

In \gls{AES}, the block size is fixed at 128 bits, but the key size can vary between 128, 192, and 256 bits. 
The key length determines the number of rounds used in the encryption process, as outlined in Table \ref{table:key-length-rounds}.

\begin{table}[h]
    \centering
    \begin{tabular}{c|c}
        \textbf{Key length (bit)} & \textbf{\# rounds ($n_r$)} \\ 
        \hline
        128 & 10 \\  
        192 & 12 \\  
        256 & 14 \\  
    \end{tabular}
    \caption{Key lengths and number of rounds for AES \cite{Paar2024}.}
    \label{table:key-length-rounds}
\end{table}
