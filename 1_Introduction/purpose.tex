\subsection{Purpose of the paper}

The purpose of this paper is to provide a comprehensive overview of \gls{AES}, one of the most widly adopted encryption algorithms used to secure digital data.
This paper aims to answer the following key questions:
\begin{itemize}
    \item What is the purpose of \gls{AES}? What security issue does it address? (Section \ref{sec:purpose})
    \item How does \gls{AES} work? (Section \ref{sec:how})
    \item Why does it work? What are the underlying mathematics surrounding the procedure?  (Section \ref{sec:why})
    \item An example to demonstrate our understanding of \gls{AES}. (Section \ref{sec:example})
\end{itemize}
Furthermore, the paper discusses the real-world applications of \gls{AES} and its limitations (Section \ref{sec:applications}). 

% The paper aims to explain the fundamental reasons for \gls{AES}'s development (Section \ref{sec:purpose}), the security challenges it addresses (Section \ref{sec:security}), and how it achieves strong encryption through its structured operations (Section \ref{sec:how}). 
% By examining the internal workings of \gls{AES}, which includes its key expansion, substitution, permutation, and mixing steps, this paper seeks to explain the encryption and decryption processes (Section \label{sec:why}).
% Furthermore, it explores the underlying mathematical principles that make \gls{AES} secure and resilient against cryptography attacks  (Section \label{sec:math}).
% To reinforce understanding, the paper also includes simplified examples demonstrating how AES operates in practice (Section \label{sec:example}). 