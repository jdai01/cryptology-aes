\section{AES High Level Structure}

\subsection{What are AES Rounds?}

AES operates on fixed-size blocks of 128 bits, and unlike the DES, it does not use a Feistel structure. 
Instead, AES performs a series of full data transformations, 
which allows it to process the entire 128-bit block in parallel during each round. 
Therefore, it shows the results in fewer rounds; however, the number of rounds depends on the key size.
\noindent
Each AES round consists of a layered structure where the internal state 
- represented as a four-by-four-byte matrix - is progressively manipulated. 
The full AES encryption process is divided into three main phases: the initial round, 
a set of main rounds, and a final round.
\noindent
The design of these rounds ensures symmetry between encryption and decryption, 
particularly due to the final round omitting the \texttt{MixColumns} transformation. 
This allows the inverse operations used during decryption to align correctly with the forward structure used in encryption.

\subsection{The Initial Round}

The initial round works through a single operation called \texttt{AddRoundKey}; in this step, 
the 128-bit plaintext block is organised into a four-by-four matrix, the state. 
Each byte of the state is combined using an XOR operation with a corresponding byte from the first-round key. 
This round key is derived from the original AES key using the key expansion process.
\noindent
Through this step, the key blends with the data from the beginning and adds an initial confusion layer, 
preparing the state for the transformations that will occur in the following rounds. 
This round does not involve any substitution or shifting steps; nevertheless, these will begin with the main rounds.

\subsection{Main Rounds}

Each main round applies the following four transformations in order:

\begin{enumerate}
    \item \textbf{SubBytes} – a nonlinear substitution step where each byte in the state is replaced through an S-box, 
    introducing non-linearity and confusion.
    \item \textbf{ShiftRows} – a permutation step that shifts the bytes in each 
    row of the state to the left by a certain number of positions, increasing diffusion.
    \item \textbf{MixColumns} – a mixing transformation that combines the bytes in each column using 
    matrix multiplication over a Galois Field, spreading the influence of each byte throughout the state.
    \item \textbf{AddRoundKey} – the updated round key is applied to the current state using the XOR operation.
\end{enumerate}
\noindent
These layers together ensure that each round produces a progressively more complex output, 
significantly transforming the state from the original input with each iteration.

\subsection{Final Round}
\noindent
The final round is similar to the main rounds, but it omits the \texttt{MixColumns} step. 
This minor simplification maintains symmetry in AES, 
allowing the decryption process to correctly reverse the encryption steps, 
as \texttt{MixColumns} is not naturally invertible.