\section{Security Analysis}

\subsection{Why AES is Considered Secure?}

AES has been widely used in practice since it was standardised by NIST more than 20 years ago. 
It is implemented in protocols and products such as TLS, IPsec, IEEE 802.11i, SSH, WhatsApp, Signal, hard disk encryption tools, and many others.\newline

\noindent AES is probably the most widely used cipher in the world today and has had a strong influence on modern block cipher design. 
Its design includes the wide trail strategy, which shows how the linear layer helps resist statistical attacks. 
Many successful modern block ciphers use design elements originally introduced in AES. 
Currently, no analytical attack that is significantly better than brute force is known on AES.

\subsection{Known Attacks on AES}

Several attacks have been proposed against AES, including the square attack, 
impossible differential attack, related-key attack, and biclique attack. 
However, none of these approaches pose a realistic threat to the full cipher.\newline

\noindent The biclique attack is the best known attack on the full AES. 
In the case of AES-128, the complexity of this attack is approximately $2^{126.1}$, 
which is only slightly better than brute force. 
Similar marginal improvements exist for AES-192 and AES-256.\newline

\noindent Elegant algebraic descriptions of AES have also been presented, 
and some authors have claimed these could be used for attacks. 
However, follow-up work has shown that these approaches are not feasible in practice.

\subsection{Why These Attacks are Hard to Pull Off?}

The biclique attack improves brute-force complexity only by a small factor and remains far from feasible in practice. 
All other known attacks either apply to reduced-round versions or require unrealistic assumptions.\newline

\noindent In the context of quantum computing, Grover’s algorithm provides a quadratic speedup over classical brute-force attacks. 
The most recent and optimised quantum circuit analyses estimate the complexity of a Grover-based attack on AES-128 as approximately $2^{156.3}$, 
on AES-192 as $2^{221.6}$, and on AES-256 as $2^{286.1}$~\cite{Jang2025}. 
These results suggest that even in a post-quantum scenario, 
AES remains secure under current technological limitations.\newline

\noindent To date, none of the proposed classical or quantum attacks have compromised the full AES cipher under realistic conditions.