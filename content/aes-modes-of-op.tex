\section{AES Modes of Operation}

\subsection{What are Modes of Operation?}

Block ciphers such as AES are designed to encrypt fixed-length blocks (e.g., 128 bits), 
but real-world applications typically require secure encryption of variable-length messages. 
Modes of operation define how to securely encrypt these longer messages by specifying how individual blocks are chained or transformed during encryption. 
They significantly influence the security and efficiency of the cipher and must be selected carefully based on application requirements.

\subsection{Electronic Codebook (ECB) Mode}

ECB is the simplest AES mode where each plaintext block is encrypted independently:
\[
y_i = E_k(x_i), \quad x_i = D_k(y_i)
\]
Its determinism leads to serious vulnerabilities: identical plaintext blocks produce identical ciphertexts. 
This exposes structural patterns in the plaintext, making it unsuitable for most practical applications.

\subsection{Cipher Block Chaining (CBC) Mode}

CBC introduces chaining by XORing each plaintext block with the previous ciphertext:
\[
\begin{aligned}
y_1 &= E_k(x_1 \oplus IV) \\
y_i &= E_k(x_i \oplus y_{i-1}) \\
x_1 &= D_k(y_1) \oplus IV \\
x_i &= D_k(y_i) \oplus y_{i-1}
\end{aligned}
\]
A random initialisation vector (IV) ensures semantic security. 
However, CBC decryption is parallelisable, while encryption is inherently sequential.

\subsection{Cipher Feedback (CFB) Mode}

CFB operates as a self-synchronising stream cipher:
\[
\begin{aligned}
s_1 &= E_k(IV), \quad y_1 = x_1 \oplus s_1 \\
s_i &= E_k(y_{i-1}), \quad y_i = x_i \oplus s_i
\end{aligned}
\]
This mode provides partial error recovery and supports smaller data units, such as bytes.

\subsection{Output Feedback (OFB) Mode}

OFB, like CFB, transforms AES into a stream cipher, but the key stream is independent of the ciphertext:
\[
\begin{aligned}
s_1 &= E_k(IV), \quad y_1 = x_1 \oplus s_1 \\
s_i &= E_k(s_{i-1}), \quad y_i = x_i \oplus s_i
\end{aligned}
\]
It offers full error isolation, but loss of synchronisation between sender and receiver leads to complete decryption failure.

\subsection{Counter (CTR) Mode}

CTR mode generates a keystream using an incrementing counter:
\[
y_i = x_i \oplus E_k(\text{Nonce} || \text{CTR}_i)
\]
It supports full parallelisation and random access to encrypted data, making it highly suitable for high-throughput systems.

\subsection{XTS-AES: A Mode for Disk Encryption}

XTS-AES was developed specifically for encrypting data on storage devices. 
It employs a tweakable block cipher and uses two keys, $k_1$ and $k_2$, 
to compute a unique tweak $T$ for each data block:

\[
\begin{aligned}
T &= E_{k_2}(i) \cdot \alpha^j \\
y_j &= E_{k_1}(x_j \oplus T) \oplus T
\end{aligned}
\]

This structure ensures that identical plaintext blocks at different locations produce different ciphertexts, and supports efficient parallel processing.

\subsection{Post-Quantum Security and AES Modes of Operation}

The development of quantum computing poses significant threats to symmetric-key ciphers via Grover’s algorithm, 
which reduces brute-force complexity from $2^k$ to about $2^{k/2}$. 

\noindent As such, AES-128 may offer only $2^{64}$ bits of effective quantum security.
Recent work by Jang et al.~\cite{Jang2025} improves resource estimation for quantum AES implementations using optimised Grover’s circuits. \newline

\noindent Their findings yield the following security estimates:
\begin{itemize}
    \item AES-128: $2^{156.26}$
    \item AES-192: $2^{221.58}$
    \item AES-256: $2^{286.07}$
\end{itemize}

The study also introduces low-depth circuit architectures designed to comply with NIST’s \texttt{MAXDEPTH} constraint, 
emphasising modes that support parallelism.

\textbf{Mode Implications:}
\begin{itemize}
    \item \textbf{ECB:} Highly insecure; reveals structure under both classical and quantum analysis.
    \item \textbf{CBC:} Limited parallelism; not ideal in depth-constrained quantum models.
    \item \textbf{CFB/OFB:} Less deterministic; may provide robustness depending on IV usage.
    \item \textbf{CTR:} Well-suited for quantum-resilient designs due to full parallelism.
    \item \textbf{XTS:} Offers random block access and tweakable security; strong candidate for post-quantum storage encryption.
\end{itemize}

\subsection{Summary of Modes of Operation}

\begin{table}[h]
\centering
\begin{tabular}{|l|c|c|c|l|}
\hline
\textbf{Mode} & \textbf{Parallel} & \textbf{Error Prop.} & \textbf{Random Access} & \textbf{Best For} \\
\hline
ECB & Yes & None & Yes & Testing, toy cases \\
CBC & No (Enc) / Yes (Dec) & Next block & No & File encryption \\
CFB & No & Next block & No & Streaming with feedback \\
OFB & Yes & None & No & Error-resilient streaming \\
CTR & Yes & None & Yes & High-speed applications \\
XTS & Yes & None & Yes & Disk encryption \\
\hline
\end{tabular}
\caption{Comparison of AES modes of operation.}
\end{table}