\subsection{Implementation details}

Building on the mathematical foundation outlined in Section~\ref{sec:why}, the AES algorithm was implemented in Python to demonstrate its practical application.
This implementation supports only AES-128, and requires both the encryption key and input data to be provided in hexadecimal format. 
The complete source code is provided in Appendix~\ref{appendix:code}.

The implementation is structured across three Python modules: \texttt{functions.py}, \texttt{bin\_math.py}, and \texttt{aes.py}, each responsible for a specific aspect of the algorithm.

The \texttt{functions.py} module (see Python Code~\ref{lst:fxns}) contains general-purpose helper functions used throughout the AES implementation. 
These functions include:
\begin{itemize}
    \item \texttt{process\_str(string)}: Removes all whitespace from a given string.
    \item \texttt{process\_key(key, Nk=4)}: Processes a hexadecimal key string into a $4\times4$ \texttt{NumPy} array of bytes.
    \item \texttt{process\_input\_hex(input\_hex)}: Processes a hexadecimal input (e.g., plaintext or ciphertext) into a $4\times4$ AES state matrix.
    \item \texttt{bytes\_to\_word(bytes\_list)}: Converts a list of 4 bytes into a 32-bit word.
    \item \texttt{word\_to\_bytes(word)}: Converts a 32-bit word into a list of 4 bytes.
    \item \texttt{state\_to\_hex(state)}: Converts a $4\times4$ AES state matrix into a hexadecimal string representation.
\end{itemize}

The \texttt{bin\_math.py} module (see Python Code~\ref{lst:bin-math}) implements the arithmetic operations required for AES's finite field calculations. 
This module includes:
\begin{itemize}
    \item \texttt{gf\_mult(a, b)}: Performs Galois Field multiplication of two bytes, as explained in Section~\ref{sec:multiplication}.
    \item \texttt{generate\_gf(primitive\_element)}: Constructs the extension field GF($2^8$) using a given primitive element, as described in Section~\ref{sec:multiplication}.
\end{itemize}

The core AES encryption logic is implemented in the \texttt{aes.py} module (see Python Code~\ref{lst:aes-impl}). 
This module includes the main algorithmic transformations and key scheduling logic, as described below:
\begin{itemize}
    \item \texttt{load\_properties(key\_length)}: Loads AES parameters such as the number of rounds based on the specified key length.
    \item \texttt{generate\_sbox()}: Generates the AES S-Box used in the \textsc{SubBytes} step (see Section~\ref{sec:SubBytes}).
    \item \texttt{generate\_rc(Nr=10)}: Produces a list of round constants used in the key expansion process (see Section~\ref{sec:key-expansion}).
    \item \texttt{g(word, rc)}: Implements the transformation function used during key expansion (see Section~\ref{sec:key-expansion}).
    \item \texttt{KeyExpansion(key, Nb=4, Nk=4, Nr=10)}: Performs AES key expansion to generate round keys (see Section~\ref{sec:key-expansion}).
    \item \texttt{enter\_key(key)}: Prepares and expands the encryption key by processing it and initializing the key expansion.
    \item \texttt{SubBytes(state)}: Performs the \textsc{SubBytes} transformation to the AES state (see Section~\ref{sec:SubBytes}).
    \item \texttt{ShiftRows(state)}: Performs the \textsc{ShiftRows} transformation (see Section~\ref{sec:ShiftRows}).
    \item \texttt{MixColumns(state)}: Performs the MixColumns transformation (see Section~\ref{sec:MixColumns}).
    \item \texttt{AddRoundKey(state, keys)}: Adds the round key to the current state using \texttt{XOR} (see Section~\ref{sec:AddRoundKey}).
    \item \texttt{encrypt(input\_)}: Executes the full AES encryption procedure on the given input.
\end{itemize}