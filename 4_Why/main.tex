\section{Why AES works? -- The mathematical foundations}
\label{sec:why}
\label{sec:math}

The 16-byte input $A_0, \dots, A_{15}$ is fed byte-wise into the S-Box in the Byte Substitution layer (Section \ref{sec:SubBytes}).
The 16-byte output $B_0, \dots, B_{15}$ is permutated twice in the \textsc{ShiftRows} (Section \ref{sec:ShiftRows}) and mixed by the \textsc{MixColumn} transformation (Section \ref{sec:MixColumns}), both in the Diffusion layer.
Finally, the 128-bit subkey $k_i$ is \texttt{XOR} with the immediate result in the Key Addition layer (Section \ref{sec:AddRoundKey}).
Figure \ref{fig:aes-round-function} shows the graph of a single AES round. 

\begin{figure}[!ht]
    \centering
    \includegraphics[width=.8\textwidth]{aes-block-diagram.png}
    \caption{
        AES Encryption Block Diagram \cite{Paar2024}.
        The plaintext is denoted as $x$, the ciphertext as $y$, key as $k$, and the number of rounds as $n_r$.
    }
    \label{fig:aes-block-diagram}
\end{figure}

The mathematical foundations discussed in subsequent subsections are based on the \gls{AES} specification published by \gls{NIST} \cite{NIST_AES} and the textbook by Paar and Pelzl \cite{Paar2024}.

\subsection{Mathematics Foundation}

As all bytes in the \gls{AES} algorithm are interpreted as finite field elements, they can be added and multiploed. 
However, these operations are different from decimal numbers and are outlined below.


\subsubsection{The Finite Field \texorpdfstring{$\mathrm{\Gls{GF}}(2^8)$}{\Gls{GF}(2^8)}}
\label{sec:galois}

The field $\mathrm{\Gls{GF}}(2^8)$ consists of 256 elements and is formed by polynomials with binary coefficients, each of degree less than 8. All arithmetic in this field is done modulo the irreducible polynomial:

\begin{align}
    m(x) = x^8 + x^4 + x^3 + x + 1 \label{eq:irred-poly}
\end{align}

Each element corresponds to an 8-bit byte. Operations in the field include:
\begin{itemize}
    \item \textbf{Addition:} Performed as bitwise XOR
    \item \textbf{Multiplication:} Polynomial multiplication modulo $m(x)$
    \item \textbf{Inversion:} Found using the Extended Euclidean Algorithm
\end{itemize}

These operations are critical in the \Gls{AES} algorithm, particularly in transformations like \texttt{SubBytes} and \texttt{MixColumns}, which depend on the structure of $\mathrm{\Gls{GF}}(2^8)$ to ensure both security and performance.

\subsubsection{Addition}
\label{sec:addition}

The addition of two elements in a finite field is perfomed with the \texttt{XOR} operation, denoted by $\oplus$, with
\begin{align}
    c_i &= a_i \oplus b_i \quad \forall i \in [0, 7]
\end{align}
in $GF(2^8)$.


\subsubsection{Multiplication}
\label{sec:multiplication}

Multiplication in $\mathrm{\Gls{GF}}(2^8)$ involves polynomial arithmetic, denoted by $\bullet$. 
Each byte is treated as a polynomial of degree at most 7 with binary coefficients. 
Multiplication is then done modulo an irreducible polynomial of degree 8, which for \Gls{AES} is described in Eq. \ref{eq:irred-poly}.

For instance, the byte $\{57\}$ (binary 01010111) represents the polynomial:

\[
x^6 + x^4 + x^2 + x + 1
\]

To multiply two bytes $a(x)$ and $b(x)$, we compute:

\begin{align}
    c(x) = a(x) \bullet b(x) \mod m(x) \label{eq:mul}
\end{align}

An example would be:

\[
\{57\} \bullet \{83\} = \{c1\}
\]

Because this operation can be computationally heavy, efficient implementations often use lookup tables or optimized routines like \texttt{xtime}, which simplifies multiplication by $x$.

\subsubsection{Multiplication by \texorpdfstring{$x$}{x}}
\label{sec:multx}

Multiplying a byte by $x$ (equivalent to $\{02\}$) involves a left shift of the byte, with an additional XOR with $\{1b\}$ if the leftmost bit (most significant bit) is 1:

\begin{align}
    \text{xtime}(b) =
    \begin{cases}
    b \ll 1, & \text{if } b_7 = 0 \\
    (b \ll 1) \oplus \{1b\}, & \text{if } b_7 = 1
    \end{cases}
\end{align}

This allows for fast multiplication by $x$ modulo the \Gls{AES} irreducible polynomial.

\subsection{\textsc{SubBytes} transformation}
\label{sec:SubBytes}

The \textsc{SubBytes} transformation is a non-linear byte substitution step that operates independently on each byte of the AES state. 
It can be viewed as the application of 16 parallel S-Boxes, each processing 8-bit input to produce an 8-bit output, as illustrated in Figure~\ref{fig:byte-substitution}. 
For every byte $A_i$ in the \gls{AES} state, the transformation produces a substituted byte $B_i$, defined by the function $B_i = S(A_i)$.

\begin{figure}[!ht] 
    \centering
    \includegraphics[width=.6\textwidth]{byte-substitution.png} 
    \caption{
        The two operations within the AES S-Box which computes the function $B_i = S(A_i)$ \cite{Paar2024}.
    }
    \label{fig:byte-substitution} 
\end{figure}

The construction of the S-Box is based on two sequential transformations:
\begin{enumerate}
    \item \textbf{Galois Field Inversion}: 
    Each byte is interpreted as an element in the finite field $GF(2^8)$ and is mapped to its multiplicative inverse in this field. 
    This operation is defined by:
    
    \begin{align}
        A_i \cdot {B'}_i &= 1 \mod m(x)
        \label{eq:gfi}
    \end{align}
    where $m(x)$ is the irreducible polynomial defining the field, as discussed in Section~\ref{sec:multiplication}. 

    The element ${00}$, which has no inverse, is mapped to itself.

    \item \textbf{Affine Mapping}:
    
    Each resulting byte ${B'}_i$ undergoes a bitwise affine transformation over $GF(2)$. 
    This transformation is defined as:
    \begin{align}
        b_i = {b'}_i \oplus {b'}_{(i+4 \mod 8)} \oplus {b'}_{(i+5 \mod 8)} \oplus {b'}_{(i+6 \mod 8)} \oplus {b'}_{(i+7 \mod 8)} \oplus c_i
    \end{align}
    for $0 \leq i \leq 8$, where $b_i$ is the $i$-th output bit of the byte, ${b'}_i$ are the bits of the inverted byte, and $c_i$ are bits from a fixed constant byte.
\end{enumerate}

A complete S-Box can be precomputed by applying these transformations to all 256 possible input bytes (from \texttt{00} to \texttt{FF}), allowing efficient lookup during AES encryption. 
The full substitution table is shown in Figure~\ref{fig:sbox}.

\begin{figure}[!ht]
    \centering
    \includegraphics[width=.8\textwidth]{sbox.png}
    \caption{S-Box: substitution values for the byte $\{xy\}$ \cite{NIST_AES}.}
    \label{fig:sbox}
\end{figure}


\paragraph{Example:} \textsc{SubByte} transformation of $(53)_{\text{hex}}$

\begin{equation}
    \textsc{SubByte}((53)_{\text{hex}}) = (ed)_{\text{hex}}
\end{equation}

\subsection{\textsc{ShiftRows}}

The \textsc{ShiftRows} transformation cyclincally shifts the second row of the state matrix three bytes to the right, the third row by two bytes to the right, and the forth row by one byte to the right (Eq. \ref{eq:ShiftRows}).
The first row is not changed.

The purpose of this is to increase the diffusion properties of AES.

\begin{equation}
    \begin{array}{c@{\quad \longrightarrow \quad}c}
        \begin{array}{|c|c|c|c|}
        \hline
        B_0 & B_4 & B_8 & B_{12} \\
        \hline
        B_1 & B_5 & B_9 & B_{13} \\
        \hline
        B_2 & B_6 & B_{10} & B_{14} \\
        \hline
        B_3 & B_7 & B_{11} & B_{15} \\
        \hline
        \end{array}
    &
    \begin{array}{|c|c|c|c|}
        \hline
        B_0 & B_4 & B_8 & B_{12} \\
        \hline
        B_5 & B_9 & B_{13} & B_1 \\
        \hline
        B_{10} & B_{14} & B_2 & B_6 \\
        \hline
        B_{15} & B_3 & B_7 & B_{11} \\
        \hline
        \end{array}
    \end{array}
    \label{eq:ShiftRows}
\end{equation}
\subsection{\textsc{MixColumns}}

The \textsc{MixColumn} transformation is a linear transformation which mixes each column of the state matrix. 
\begin{align}
    MixColumn(B) &= C\\
    \begin{pmatrix}
        02 & 03 & 01 & 01\\
        01 & 02 & 03 & 01\\
        01 & 01 & 02 & 03\\
        03 & 01 & 01 & 02
    \end{pmatrix}
    \cdot
    \begin{pmatrix}
        B_{0,c} \\
        B_{1,c} \\
        B_{2,c} \\
        B_{3,c} \\
    \end{pmatrix}
    &=
    \begin{pmatrix}
        C_{0,c} \\
        C_{1,c} \\
        C_{2,c} \\
        C_{3,c} \\
    \end{pmatrix}
\end{align}
for $0 \leq c \leq Nb$.
% with $B$ being the 16-byte input state after \texttt{ShiftRows} operation given in Equation \ref{} and C being the 16-byte output state.
% Each vector column of B is multiplied by a fixed $4 \times 4$ matrix (containing constant entries).

As a result of this multiplication, the four bytes in a column are replaced with the following
\begin{align}
    C_{0,c} & = (\{02\} * B_{0,c}) &&\oplus (\{03\} * B_{1,c}) &&\oplus B_{2,c} &&\oplus B_{3,c}\\
    C_{1,c} & = B_{0,c} &&\oplus (\{02\} * B_{1,c}) &&\oplus (\{03\} * B_{2,c}) &&\oplus B_{3,c}\\
    C_{2,c} & = B_{0,c}  &&\oplus B_{1,c} &&\oplus (\{02\} * B_{2,c}) &&\oplus (\{03\} * B_{3,c})\\
    C_{3,c} & = (\{03\} * B_{0,c}) &&\oplus B_{1,c} &&\oplus B_{2,c} &&\oplus (\{02\} * B_{3,c})
\end{align}

\subsection{\textsc{AddRoundKey} transformation}
\label{sec:AddRoundKey}

The \textsc{AddRoundKey} transformation combines the current state matrix with a round key derived from the key expansion process (Section \ref{sec:key-expansion}). 

This transformation operates by performing a bitwise \texttt{XOR} operation between each byte of the state matrix and the corresponding byte of the round key, represented as:
\begin{equation}
    C_{i,\text{out}} = C_{i,\text{in}} \oplus K_i
\end{equation}
where $ C_{i,\text{in}}$ and $C_{i,\text{out}}$ denote the input and output bytes of the state respectively, and $K_i$ represents the corresponding byte of the round key.

This step introduces key-dependent confusion into the encryption process, ensuring that the state is uniquely modified at each round based on the key material.
\subsection{Key Expansion (Key Schedule)}


\subsection{Security Properties}

Modes of operation define how a block cipher like AES is applied to variable-length data. 
Each mode introduces distinct security implications, 
including confidentiality, error propagation, resistance to structural analysis, 
and resilience against future threats such as quantum computing.

\noindent The security of a mode depends not only on AES itself but also on how the blocks are processed. 
For instance:

\begin{itemize}
    \item \textbf{\Gls{ECB} Mode} encrypts each block independently, leading to pattern leakage. Identical plaintext blocks produce identical ciphertexts, compromising confidentiality, especially in structured data like images.
    
    \item \textbf{\Gls{CBC} Mode} introduces randomisation through an Initialisation Vector (IV), ensuring that identical messages yield different ciphertexts. However, encryption is inherently sequential and more sensitive to bit-flip propagation.
    
    \item \textbf{\Gls{CFB} and \Gls{OFB} Modes} convert the block cipher into a stream cipher. OFB offers better error isolation, while CFB is self-synchronising but more susceptible to feedback-based error propagation.
    
    \item \textbf{\Gls{CTR} Mode} enables full parallelism and random access by encrypting incrementing counters. It avoids feedback loops and supports high-throughput use cases, making it robust in environments requiring speed and scalability.
    
    \item \textbf{\Gls{XTS}-AES} is optimised explicitly for disk encryption. It uses a tweakable block cipher based on sector position to ensure that the same data encrypted at different locations yields different ciphertexts, offering strong structural and positional integrity.
\end{itemize}

With the anticipated rise of quantum computing, 
AES modes must also be evaluated under quantum threat models. 
Grover’s algorithm reduces brute-force complexity from $2^k$ to approximately $2^{k/2}$. \newline

Jang et al.\cite{Jang2025} present refined circuit depth and gate count estimates under realistic quantum constraints. 
Their study indicates the following effective quantum security levels:

\begin{itemize}
    \item AES-128: $2^{156.26}$
    \item AES-192: $2^{221.58}$
    \item AES-256: $2^{286.07}$
\end{itemize} 

These values, derived using optimised Grover-based search circuits, 
are well above the estimated practical capabilities of near-term quantum computers\cite{Jang2025}. \newline

Importantly, parallelisable modes like CTR and XTS are more amenable to low-depth quantum circuit implementations. 
Their compatibility with constraints such as NIST’s \texttt{MAXDEPTH} parameter makes them promising candidates for quantum-resilient applications\cite{Jang2025}.

\begin{table}[h]
\centering
\begin{tabular}{|l|c|c|c|l|}
\hline
\textbf{Mode} & \textbf{Parallel} & \textbf{Error Propagation} & \textbf{Random Access} & \textbf{Best Use Case} \\
\hline
\Gls{ECB} & Yes & None & Yes & Toy cases, testing only \\
\Gls{CBC} & No (Enc) / Yes (Dec) & Next block & No & File encryption \\
\Gls{CFB} & No & Next block & No & Streaming data with feedback \\
\Gls{OFB} & Yes & None & No & Resilient byte-wise streaming \\
\Gls{CTR} & Yes & None & Yes & High-throughput systems \\
\Gls{XTS} & Yes & None & Yes & Secure disk encryption \\
\hline
\end{tabular}
\caption{Security characteristics of AES modes of operation.}
\end{table}
\newpage

