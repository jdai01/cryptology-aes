\subsection{Mathematics Foundation}

As all bytes in the \gls{AES} algorithm are interpreted as finite field elements, they can be added and multiploed. 
However, these operations are different from decimal numbers and are outlined below.


\subsubsection{The Finite Field \texorpdfstring{$\mathrm{\Gls{GF}}(2^8)$}{\Gls{GF}(2^8)}}
\label{sec:galois}

The field $\mathrm{\Gls{GF}}(2^8)$ consists of 256 elements and is formed by polynomials with binary coefficients, each of degree less than 8. All arithmetic in this field is done modulo the irreducible polynomial:

\begin{align}
    m(x) = x^8 + x^4 + x^3 + x + 1 \label{eq:irred-poly}
\end{align}

Each element corresponds to an 8-bit byte. Operations in the field include:
\begin{itemize}
    \item \textbf{Addition:} Performed as bitwise XOR
    \item \textbf{Multiplication:} Polynomial multiplication modulo $m(x)$
    \item \textbf{Inversion:} Found using the Extended Euclidean Algorithm
\end{itemize}

These operations are critical in the \Gls{AES} algorithm, particularly in transformations like \texttt{SubBytes} and \texttt{MixColumns}, which depend on the structure of $\mathrm{\Gls{GF}}(2^8)$ to ensure both security and performance.

\subsubsection{Addition}
\label{sec:addition}

The addition of two elements in a finite field $GF(2^8)$ is perfomed with the bitwise \texttt{XOR} operation, denoted by $\oplus$.
Formally, this can be expressed as
\begin{align}
    c_i &= a_i \oplus b_i \quad \forall i \in [0, 7]
\end{align}
where $a_i, b_i$, and $c_i$ represents the bits of the input elements and the result respectively.


\paragraph{Example:} Addition of $(57)_{\text{hex}}$ and $(83)_{\text{hex}}$
\[
\begin{array}{ccccccccccc}
     & 0 & 1 & 0 & 1 & 0 & 1 & 1 & 1 & = & \{57\} \\
\oplus & 1 & 0 & 0 & 0 & 0 & 0 & 1 & 1 & = & \{83\} \\
\hline
     & 1 & 1 & 0 & 1 & 0 & 1 & 0 & 0 & = & \{d4\} \\
\end{array}
\]

Therefore, $(57)_{\text{hex}} \oplus (83)_{\text{hex}} = (d4)_{\text{hex}}$. 


\subsubsection{Multiplication}
\label{sec:multiplication}

Multiplication in a finite field $GF(2^8)$, denoted by $\bullet$, is defined as the multiplication of two polynomials modulo an irreducible polynomial of degree 8.
Formally, the operation is expressed as:
\begin{equation}
    c(x) = a(x) \bullet b(x) \mod m(x)
    \label{eq:mul}
\end{equation}
where $a(x)$ and $b(x)$ are polynomials representing the field elements, and $m(x)$ is the reducible polynomial defining the field.

For \gls{AES} algorithm, this irreducible polynomial $m(x)$ is fixed and given as follow:
\begin{equation}
    m(x) = x^8 + x^4 + x^3 + x + 1
\end{equation}
This polynomial corresponds to the binary representation $(100011010)_{\text{bin}}$.

\paragraph{Example:} Multiplication of $(53)_{\text{hex}}$ and $(83)_{\text{hex}}$

The polynomial multiplication before modulo reduction is
\[
\begin{array}{ccl}
    \overbrace{(x^6 + x^4 + x^2 + x^1 + 1)}^{(53)_{\text{hex}}} \bullet \overbrace{(x^7 + x + 1)}^{(83)_{\text{hex}}} \mod m(x) & = & \left(x^{13} + x^{11} + x^9 + x^8 + x^7\\
    && + x^7 + x^5 + x^3 + x^2 + x\\
    && + x^6 + x^4 + x^2 + x + 1 \right) \mod m(x)
\end{array}
\]

The resulting polynomial sum is computed using the bitwise \texttt{XOR} operation over the coefficients:
\[
\begin{array}{ccccccccccccccc}
    &1&0&1&0&1&1&1&0&0&0&0&0&0&0\\
    \oplus &&&&&&&1&0&1&0&1&1&1&0\\
    \oplus &&&&&&&&1&0&1&0&1&1&1\\
    \hline
    &1&0&1&0&1&1&0&1&1&1&1&0&0&1
\end{array}
\]

Next, the modulo operation by the irreducible polynomial $m(x)$ is performed:
\[
\begin{array}{cccccccccccccccc}
    &1&0&1&0&1&1&0&1&1&1&1&0&0&1\\
    \oplus &1&0&0&0&1&1&0&1&0\\
    \hline
    &0&0&1&0&0&0&0&0&0&1&1&0&0&1\\
    \oplus &&&1&0&0&0&1&1&0&1&0\\
    \hline
    &&&0&0&0&0&1&1&0&0&0&0&0&1
\end{array}
\].

The final result corresponds to the binary value $(11000001)_{\text{bin}} = (c1)_{\text{hex}}$.

Thus, $(57)_{\text{hex}} \bullet (83)_{\text{hex}} = (c1)_{\text{hex}}$.


% Because this operation can be computationally heavy, efficient implementations often use lookup tables or optimized routines like \texttt{xtime}, which simplifies multiplication by $x$.

\subsubsection{Multiplication by \texorpdfstring{$x$}{x}}
\label{sec:multx}

Multiplying a byte by $x$ (equivalent to $\{02\}$) involves a left shift of the byte, with an additional XOR with $\{1b\}$ if the leftmost bit (most significant bit) is 1:

\begin{align}
    \text{xtime}(b) =
    \begin{cases}
    b \ll 1, & \text{if } b_7 = 0 \\
    (b \ll 1) \oplus \{1b\}, & \text{if } b_7 = 1
    \end{cases}
\end{align}

This allows for fast multiplication by $x$ modulo the \Gls{AES} irreducible polynomial.
