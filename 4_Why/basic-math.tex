\subsection{Mathematical Preliminaries}

\Gls{AES} relies heavily on operations over a specific mathematical structure known as the finite field $\mathrm{\Gls{GF}}(2^8)$. In this section, we explore the core arithmetic concepts within this field that are fundamental to how AES works.

\subsubsection{Addition}
\label{sec:addition}

In $\mathrm{\Gls{GF}}(2^8)$, addition is performed using bitwise \Gls{XOR}, which essentially means each bit of the result is the modulo 2 sum of the corresponding bits of the inputs. If we have two bytes:

\[
a = \{a_7a_6a_5a_4a_3a_2a_1a_0\}, \quad b = \{b_7b_6b_5b_4b_3b_2b_1b_0\}
\]

then their sum $c$ is calculated as:

\[
c_i = a_i \oplus b_i \quad \text{for } i = 0 \text{ to } 7
\]

Or more simply:

\[
c = a \oplus b
\]

For example:

\[
\{57\} \oplus \{83\} = \{d4\}
\]

This operation is equivalent to adding polynomials over $\mathrm{\Gls{GF}}(2)$ (binary coefficients) where addition and subtraction are identical since both are performed modulo 2.

\subsubsection{Multiplication}
\label{sec:multiplication}

Multiplication in $\mathrm{\Gls{GF}}(2^8)$ involves polynomial arithmetic. Each byte is treated as a polynomial of degree at most 7 with binary coefficients. Multiplication is then done modulo an irreducible polynomial of degree 8, which for \Gls{AES} is:

\[
m(x) = x^8 + x^4 + x^3 + x + 1
\]

For instance, the byte $\{57\}$ (binary 01010111) represents the polynomial:

\[
x^6 + x^4 + x^2 + x + 1
\]

To multiply two bytes $a(x)$ and $b(x)$, we compute:

\[
c(x) = a(x) \cdot b(x) \mod m(x)
\]

An example would be:

\[
\{57\} \cdot \{83\} = \{c1\}
\]

Because this operation can be computationally heavy, efficient implementations often use lookup tables or optimized routines like \texttt{xtime}, which simplifies multiplication by $x$.

\subsubsection{Multiplication by \texorpdfstring{$x$}{x}}
\label{sec:multx}

Multiplying a byte by $x$ (equivalent to $\{02\}$) involves a left shift of the byte, with an additional XOR with $\{1b\}$ if the leftmost bit (most significant bit) is 1:

\[
\text{xtime}(b) =
\begin{cases}
b \ll 1, & \text{if } b_7 = 0 \\
(b \ll 1) \oplus \{1b\}, & \text{if } b_7 = 1
\end{cases}
\]

This allows for fast multiplication by $x$ modulo the \Gls{AES} irreducible polynomial.

\subsubsection{The Finite Field \texorpdfstring{$\mathrm{\Gls{GF}}(2^8)$}{\Gls{GF}(2^8)}}
\label{sec:galois}

The field $\mathrm{\Gls{GF}}(2^8)$ consists of 256 elements and is formed by polynomials with binary coefficients, each of degree less than 8. All arithmetic in this field is done modulo the irreducible polynomial:

\[
m(x) = x^8 + x^4 + x^3 + x + 1
\]

Each element corresponds to an 8-bit byte. Operations in the field include:
\begin{itemize}
    \item \textbf{Addition:} Performed as bitwise XOR
    \item \textbf{Multiplication:} Polynomial multiplication modulo $m(x)$
    \item \textbf{Inversion:} Found using the Extended Euclidean Algorithm
\end{itemize}

These operations are critical in the \Gls{AES} algorithm, particularly in transformations like \texttt{SubBytes} and \texttt{MixColumns}, which depend on the structure of $\mathrm{\Gls{GF}}(2^8)$ to ensure both security and performance.