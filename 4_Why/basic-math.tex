\subsection{Mathematics foundations}

As all bytes in the \gls{AES} algorithm are treated as elements of a finite field, operations such as addition and multiplication are performed according to the rules of the Galois Field $GF(2^8)$, rather than standard arithmetic. 
These operations use polynomial representations and are essential to \gls{AES} transformations.


\subsubsection{Galois Field $GF(2^8)$}
\label{sec:gf}

A Galois Field is a field that contains a finite number of elements, denoted as
\begin{equation}
    GF(p^n)
\end{equation}
where $p$ is a prime number (the characteristic of the field), and $n \in \mathbb{N}$ (the degree of extension).

In the AES algorithm, $GF(2^8)$ is used as its elements are 8-bit binary numbers (ranging from 0 to 255), and it supports both addition and multiplication operations fundamental for the algorithm's transformations.
The bytes can be interpreted as polynomial representation:
\begin{equation}
    b_7 x^7 + b_6 x^6 + b_5 x^5 + b_4 x^4 + b_3 x^3 + b_2 x^2 + b_1 x^1 = \sum_{i=0}^7 b_i x^i \text{ with } b_i \in \{0,1\}
    \label{eq:gf2^8}
\end{equation}.


\subsubsection{Addition}
\label{sec:addition}

The addition of two elements in a finite field $GF(2^8)$ is performed with the bitwise \texttt{XOR} operation, denoted by $\oplus$.
Formally, this can be expressed as
\begin{align}
    c_i &= a_i \oplus b_i \quad \forall i \in [0, 7]
\end{align}
where $a_i, b_i$, and $c_i$ represents the bits of the input elements and the result respectively.


\paragraph{Example:} Addition of $(57)_{\text{hex}}$ and $(83)_{\text{hex}}$
\[
\begin{array}{ccccccccccc}
     & 0 & 1 & 0 & 1 & 0 & 1 & 1 & 1 & = & \{57\} \\
\oplus & 1 & 0 & 0 & 0 & 0 & 0 & 1 & 1 & = & \{83\} \\
\hline
     & 1 & 1 & 0 & 1 & 0 & 1 & 0 & 0 & = & \{d4\} \\
\end{array}
\]

Therefore, $(57)_{\text{hex}} \oplus (83)_{\text{hex}} = (d4)_{\text{hex}}$. 


\subsubsection{Multiplication}
\label{sec:multiplication}

Multiplication in a finite field $GF(2^8)$, denoted by $\bullet$, is defined as the multiplication of two polynomials modulo an irreducible polynomial of degree 8.
Formally, the operation is expressed as:
\begin{equation}
    c(x) = a(x) \bullet b(x) \mod m(x)
    \label{eq:mul}
\end{equation}
where $a(x)$ and $b(x)$ are polynomials representing the field elements, and $m(x)$ is the reducible polynomial defining the field.

For \gls{AES} algorithm, this irreducible polynomial $m(x)$ is fixed and given as follows:
\begin{equation}
    m(x) = x^8 + x^4 + x^3 + x + 1
\end{equation}
This polynomial corresponds to the binary representation $(100011010)_{\text{bin}}$.

\paragraph{Example:} Multiplication of $(53)_{\text{hex}}$ and $(83)_{\text{hex}}$

The polynomial multiplication before modulo reduction is
\[
\begin{array}{ccl}
    \overbrace{(x^6 + x^4 + x^2 + x^1 + 1)}^{(53)_{\text{hex}}} \bullet \overbrace{(x^7 + x + 1)}^{(83)_{\text{hex}}} \mod m(x) & = & (x^{13} + x^{11} + x^9 + x^8 + x^7\\
    && + x^7 + x^5 + x^3 + x^2 + x\\
    && + x^6 + x^4 + x^2 + x + 1 ) \mod m(x)
\end{array}
\]

The resulting polynomial sum is computed using the bitwise \texttt{XOR} operation over the coefficients:
\[
\begin{array}{ccccccccccccccc}
    &1&0&1&0&1&1&1&0&0&0&0&0&0&0\\
    \oplus &&&&&&&1&0&1&0&1&1&1&0\\
    \oplus &&&&&&&&1&0&1&0&1&1&1\\
    \hline
    &1&0&1&0&1&1&0&1&1&1&1&0&0&1
\end{array}
\]

Next, the modulo operation by the irreducible polynomial $m(x)$ is performed:
\[
\begin{array}{cccccccccccccccc}
    &1&0&1&0&1&1&0&1&1&1&1&0&0&1\\
    \oplus &1&0&0&0&1&1&0&1&0\\
    \hline
    &0&0&1&0&0&0&0&0&0&1&1&0&0&1\\
    \oplus &&&1&0&0&0&1&1&0&1&0\\
    \hline
    &&&0&0&0&0&1&1&0&0&0&0&0&1
\end{array}
\].

The final result corresponds to the binary value $(11000001)_{\text{bin}} = (c1)_{\text{hex}}$.

Thus, $(57)_{\text{hex}} \bullet (83)_{\text{hex}} = (c1)_{\text{hex}}$.