\subsection{Security properties}

Modes of operation define how a block cipher like AES is applied to variable-length data. 
Each mode introduces distinct security implications, 
including confidentiality, error propagation, resistance to structural analysis, 
and resilience against future threats such as quantum computing.

The security of a mode depends not only on AES itself but also on how the blocks are processed. 
For instance:

\begin{itemize}
    \item \textbf{\Gls{ECB} Mode} encrypts each block independently, leading to pattern leakage. Identical plaintext blocks produce identical ciphertexts, compromising confidentiality, especially in structured data like images.
    
    \item \textbf{\Gls{CBC} Mode} introduces randomisation through an \Gls{IV}, ensuring that identical messages yield different ciphertexts. However, encryption is inherently sequential and more sensitive to bit-flip propagation.
    
    \item \textbf{\Gls{CFB} and \Gls{OFB} Modes} convert the block cipher into a stream cipher. OFB offers better error isolation, while CFB is self-synchronising but more susceptible to feedback-based error propagation.
    
    \item \textbf{\Gls{CTR} Mode} enables full parallelism and random access by encrypting incrementing counters. It avoids feedback loops and supports high-throughput use cases, making it robust in environments requiring speed and scalability.
    
    \item \textbf{\Gls{XTS}-AES} is optimised explicitly for disk encryption. It uses a tweakable block cipher based on sector position to ensure that the same data encrypted at different locations yields different ciphertexts, offering strong structural and positional integrity.
\end{itemize}

With the anticipated rise of quantum computing, 
AES modes must also be evaluated under quantum threat models. 
Grover's algorithm reduces brute-force complexity from $2^k$ to approximately $2^{k/2}$. 

Jang et al. \cite{Jang2025} present refined circuit depth and gate count estimates under realistic quantum constraints. 
Their study indicates the following effective quantum security levels:

\begin{itemize}
    \item AES-128: $2^{156.26}$
    \item AES-192: $2^{221.58}$
    \item AES-256: $2^{286.07}$
\end{itemize} 

These values, derived using optimised Grover-based search circuits, 
are well above the estimated practical capabilities of near-term quantum computers \cite{Jang2025}. 

Importantly, parallelisable modes like CTR and XTS are more amenable to low-depth quantum circuit implementations. 
Their compatibility with constraints such as NIST's \texttt{MAXDEPTH} parameter makes them promising candidates for quantum-resilient applications \cite{Jang2025}.

\begin{table}[h]
\centering
\resizebox{\textwidth}{!}{%
\begin{tabular}{|l|c|c|c|l|}
\hline
\textbf{Mode} & \textbf{Parallel} & \textbf{Error Propagation} & \textbf{Random Access} & \textbf{Best Use Case} \\
\hline
\Gls{ECB} & Yes & None & Yes & Toy cases, testing only \\
\Gls{CBC} & No (Enc) / Yes (Dec) & Yes (next block) & No & File encryption \\
\Gls{CFB} & No & Yes (next block) & No & Streaming data with feedback \\
\Gls{OFB} & Yes & None & No & Resilient byte-wise streaming \\
\Gls{CTR} & Yes & None & Yes & High-throughput systems \\
\Gls{XTS} & Yes & None & Yes & Secure disk encryption \\
\hline
\end{tabular}%
}
\caption{Security characteristics of AES modes of operation.}
\end{table}