\subsection{Key Expansion}
\label{sec:key-expansion}

The key expansion algorithm in \gls{AES} is responsible for generating the round keys from the original cipher key. 
These round keys are required at each round of the encryption and decryption processes. 
For an $N_r$-round \gls{AES} cipher, a total of $N_r + 1$ round keys are needed (refer to Table~\ref{table:key-length-rounds}).

AES employs a word-oriented key schedule, where one word consists of 32 bits (4 bytes).
The generated round keys are stored in a one-dimensional key expansion array $W$, which holds all the words produced during the expansion process.

In AES-128, the cipher key is 128 bits in length, which corresponds to 4 words. 
The key expansion produces 11 round keys, amounting to a total of 44 words ($W[0], \dots, W[43]$) (Figure \ref{fig:key-schedule-128}).
The initial key forms the first four words of the key expansion array:
\begin{equation}
    W[0], W[1], W[2], W[3] = \text{Original AES Key}
\end{equation}

For each subsequent group of 4 words, the first word is computed using a non-linear transformation function \texttt{g()}, and the remaining words are generated through a recursive \texttt{XOR} operation:
\begin{align}
    W[4i] &= W[4(i-1)] \oplus g(W[4i-1])\\
    W[4i+j] &= W[4i+j-i] \oplus W[4(i-1)+j]
\end{align}
where $i = 1,\dots,10$ corresponds to the round index and $j=1,2,3$ correspond to the word index of round key at $i$.

The function \texttt{g()} introduces non-linearity and diffusion into the key expansion process. 
It operates on a single 4-byte word and comprises three steps:
\begin{enumerate}
    \item \textbf{RotWord}: Performs a cyclic left shift on the input word by one byte.
    \begin{align}
        \begin{pmatrix}
            B_0, B_1, B_2, B_3
        \end{pmatrix}
        \rightarrow
        \begin{pmatrix}
            B_1, B_2, B_3, B_0
        \end{pmatrix}
    \end{align}

    \item \textbf{SubWord}: Applies the \textsc{SubBytes} transformation to each byte of the word, substituting them using the AES S-Box (see Section \ref{sec:SubBytes}).
    
    \item \textbf{Round Constant (Rcon)}: Adds a round-dependent constant to the most significant byte of the word. 
    $\text{Rcon}[i]$ is defined as:
    \begin{equation}
        \text{Rcon}[i] = (r_i, 0, 0, 0)
    \end{equation}
    where $r_i = 2^{i-1}$ is computed in the finite field $GF(2^8)$
\end{enumerate}

The transformation \texttt{g()} plays a crucial role in strengthening the cipher by adding confusion and preventing linear relationships between the round keys and the original cipher key.

\begin{figure}[!ht]
    \centering
    \includegraphics[width=\textwidth]{img/key-schedule-128.png}
    \caption{Key schedule for AES-128 \cite{Paar2024}.}
    \label{fig:key-schedule-128}
\end{figure}

AES-192 and AES-256 differ from AES-128 primarily in terms of their key length and the number of rounds executed during encryption.
For AES-192, the key expansion process generates 13 round keys, totaling 52 words (Figure \ref{fig:key-schedule-192}). 
The computation of the key expansion array elements follows a procedure similar to AES-128; however, each iteration produces 6 new words instead of 4.
In the case of AES-256, 15  round keys are generated, amounting to 60 words (Figure \ref{fig:key-schedule-256}). 
Each iteration of the key expansion calculates 8 words. 
Additionally, AES-256 employs an extra function, denoted as \texttt{h()}, which performs a \textsc{SubByte} transformation between the third and fourth words of each round iteration, thereby increasing the complexity of the key schedule.

\begin{figure}[!ht]
    \centering
    \includegraphics[width=\textwidth]{img/key-schedule-192.png}
    \caption{Key schedule for AES-192 \cite{Paar2024}.}
    \label{fig:key-schedule-192}
\end{figure}

\begin{figure}[!ht]
    \centering
    \includegraphics[width=\textwidth]{img/key-schedule-256.png}
    \caption{Key schedule for AES-256 \cite{Paar2024}.}
    \label{fig:key-schedule-256}
\end{figure}